\documentclass[12pt, letterpaper]{article}
\usepackage[utf8]{inputenc}

\title{CMSC499A: Mutation Visualization}
\author{Mark Keller \thanks{mentored by Professor Max Leiserson}}
\date{Spring 2018}

\begin{document}
\maketitle

\begin{abstract}
Identification and analysis of patterns in data can be difficult without visualization tools. 
Datasets of somatic mutations in cancer are no exception. 
Recent whole-genome sequencing projects such as PanCancer Analysis of Whole Genomes (PCAWG) have produced large amounts of data to be explored.
Of great importance is the classification of different mutational processes and the mutational signatures they leave behind.
As new mutational signatures continue to be discovered, observation of the levels of signature activity, called signature exposure, helps show how the underlying mutational processes differ across cancer types, time, and environmental variables.
\end{abstract}

\section{Introduction}
Slight modifications to data sources used in past studies of mutations and mutation signatures, such as expansion to more cancer types, have the potential to solidify or weaken conclusions made based on limited data.
Web-based interactive visualizations allow for comparison across data sets and features.
Users can choose sequencing project data sets, mutation signature combinations (of which can be selected based on cancer-type-specific presets), and plot types. 

\end{document}
